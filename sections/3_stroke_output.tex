%%%%%%%%%%%%%%%%%%%%% Stroke output %%%%%%%%%%%%%%%%%%%%%%%%%%%%%%%%%%%%%

\section{Stroke output}

\subsection{Stroke peer-reviewed papers}

\begin{itemize}
    \item Pearn K, Allen M, Laws A, Monks T, Everson R, James M. (2023) What would other emergency stroke teams do? Using explainable machine learning to understand variation in thrombolysis practice. \textit{European Stroke Journal}. \url{https://doi.org/10.1177/23969873231189040}

    \item James, C., Allen, M,. James, M., Everson, R. (2023) Using machine learning and clinical registry data to uncover variation in clinical decision making. \textit{Intelligence-Based Medicine}. 7, 100098.  \url{https://doi.org/10.1016/j.ibmed.2023.100098.}

    \item Allen, M. James, C., Frost,J., Liabo, K., Pearn, K., Monks, T., Everson, R., Stein, K. and James, M.  (2022). Use of Clinical Pathway Simulation and Machine Learning to Identify Key Levers for Maximizing the Benefit of Intravenous Thrombolysis in Acute Stroke.  \textit{Stroke} 53:2758–2767 \url{https://doi.org/10.1161/STROKEAHA.121.038454}. 

    \item Allen, M., James, C., Frost, J. Liabo, K., Pearn, K., Monks, T., Zhelev, Z., Logan, S., Everson, R., James, M., Stein, K. (2022). Using simulation and machine learning to maximise the benefit of intravenous thrombolysis in acute stroke in England and Wales: the SAMueL modelling and qualitative study. \textit{Health and Social Care Delivery Research} 10(31):1–148. \url{https://doi.org/10.3310/GVZL5699} Accompanying online book: Available online at: \url{https://samuel-book.github.io/samuel-1/}.
    
    \item Allen M, Pearn K, Ford GA, White P, Rudd AG, McMeekin P, Stein K, James M. (2022) National implementation of reperfusion for acute ischaemic stroke in England: How should services be configured? A modelling study. \textit{European Stroke Journal} 7:28-40. \url{https://doi.org/10.1177/23969873211063323}

    \item McMeekin, P., Flynn, D., Allen, M., Coughlan, D., Ford, G.A., Lumley, H., Joyce, B.S., James, M.A., Stein, K., Burgess, B. and White, P. (2019) Estimating the effectiveness and cost-effectiveness of establishing additional endovascular Thrombectomy stroke Centres in England: a discrete event simulation. BMC Health Services Research 2019;19:821. \url{https://bmchealthservres.biomedcentral.com/articles/10.1186/s12913-019-4678-9}

    \item Allen, M., Pearn, K., Monks, T., Bray, B., Everson, R., Salmon, A., James, M. and Stein, K. (2019) Can clinical audits be enhanced by pathway simulation and machine learning? An example from the acute stroke pathway. BMJ Open 2019;9:e028296. \url{https://bmjopen.bmj.com/content/9/9/e028296}

    \item Peulter, A, Redekop, K., Allen, M, Peters, J., Eker, O. \& Severnes, J. (2019). Exploring the cost-effectiveness of mechanical thrombectomy beyond six hours following advanced imaging in the UK. Stroke 50(11):3220-3227. \url{https://www.ahajournals.org/doi/10.1161/STROKEAHA.119.026816}

    \item Allen, M., Pearn, K.,Villeneuve, E., James, M., Stein, K. (2019). Planning and providing acute stroke care in England: The effect of planning footprint size. Frontiers in Neurology. \url{https://doi.org/10.3389/fneur.2019.00150}

    \item Allen, M., Pearn, K., James, M., Ford, G.A., White, P. Rudd, A.G., McMeekin, P. \& Stein, K. (2018). Maximising access to thrombectomy services for stroke in England: a modelling study. Eur. Stroke. J. \url{http://journals.sagepub.com/doi/10.1177/2396987318785421}

    \item Allen, M., Pearn, K., Villeneuve, E., Monks, T. Stein, K. \& James, M. (2017) Feasibility of a hyper-acute stroke unit model of care across England: a modelling analysis. BMJ Open. \url{http://dx.doi.org/10.1136/bmjopen-2017-018143}

    \item Monks, T., Van der Zee, D-J., Lahe, M., Allen, M., Pearn, M., James, M., Buskens, E., Luijckxc, G-J. (2017) A framework to accelerate simulation studies of hyperacute stroke systems. Operations Research For Health Care 15:57-67. \url{https://doi.org/10.1016/j.orhc.2017.09.002}

\end{itemize}


\subsection{Stroke non-peer-reviewed publications (medRxiv, etc.)}

\begin{itemize}
    \item Allen, M., Pearn, K., Stein, K, and James, M. (2020) Estimation of stroke outcomes based on time to thrombolysis and thrombectomy. medRxiv doi:10.1101/2020.07.18.20156653 \url{https://www.medrxiv.org/content/10.1101/2020.07.18.20156653v1}
\end{itemize}

\subsection{Stroke conference presentations}

\begin{itemize}
    \item Pearn, K., Allen, M., Laws, A., Everson, R. \& James, M. (2023). What would other emergency stroke teams do? Using explainable machine learning to understand variation in thrombolysis practice. Accepted for UK Stroke Forum, December 2023.
    
    \item Pearn, K., Allen, M., Laws, A., Everson, R. \& James, M. (2023). What would other emergency stroke teams do? Using explainable machine learning to understand variation in thrombolysis practice. European Stroke Organisation Conference. Zenodo. \url{https://doi.org/10.5281/zenodo.7878306}

    \item Pearn, K., Allen, M., Laws, A., Everson, R. \& James, M. (2023). What would other emergency stroke teams do? Using explainable machine learning to understand variation in thrombolysis (clot-busting) practice. AI UK 2023 (Turing Institute), London, UK. Zenodo. \url{https://doi.org/10.5281/zenodo.7759059}

    \item Allen, M., Pearn, K, Pitt, M., Monks, T., Stein, K., James, M. (2016) Factors driving thrombolysis use and speed. A retrospective data analysis and modelling study of seven acute hospitals. European Stroke Organisation Conference 2016. 

    \item Allen, M., Pearn, K, Pitt, M., Monks, T., Stein, K., James, M. (2016) Centralising stroke services in a mixed urban and rural environment: winners and losers? European Stroke Organisation Conference 2016.

    \item Pearn, K, Allen, M., Pitt, M., Monks, T., Stein, K., James, M. (2016) ‘Ring-fencing’ beds for stroke patients in acute and community settings. Is it feasible to reserve stroke beds for stroke patients? European Stroke Organisation Conference 2016. 

    \item Pearn, K, Allen, M., Pitt, M., Monks, T., Stein, K., James, M. (2016) Which way should the ambulance turn? Is time to hospital arrival or time to treatment most important? European Stroke Organisation Conference 2016. 

    \item Pearn, K, Allen, M., Pitt, M., Monks, T., Stein, K., James, M. (2016) Use of modelling to plan regional re-configuration of hyper-acute stroke services. European Stroke Organisation Conference 2016.

\end{itemize}

\subsection{Impact, and highlighting of work}

\textbf{Geographical Modelling}

\begin{itemize}

    \item We have provided commissioned analyses and advice to planning stroke services in:

    \begin{itemize}
        \item England (NHS England)
        \item Scotland (NHS Scotland)
        \item Wales (NHS Wales)
        \item Northern Ireland (HSC: Health and Social Care in Northern Ireland
        \item South West England (South West Cardiac Strategic Clinical Network)
        \item East of England (NHS England – East of England Region)
        \item London (NHS England - London Region)
    \end{itemize}

    \item Geographic modelling used for the UK thrombectomy implementation guide: Michael Allen, Kerry Pearn, Martin James, Phil White and Ken Stein (2022). How many comprehensive and acute stroke centres should the UK have? In: Mechanical thrombectomy for acute ischaemic stroke: an implementation guide for the UK. \url{https://www.oxfordahsn.org/wp-content/uploads/2022/02/Mechanical-Thrombectomy-for-Ischaemic-Stroke-February-2022.pdf}

\end{itemize}

\textbf{SAMueL}

\begin{itemize}
    \item NIHR have highlighted SAMueL-1 at \url{https://evidence.nihr.ac.uk/alert/increasing-thrombolysis-use-after-stroke-lessons-from-machine-learning/}. 
    \item \textit{Stroke} wrote an editorial on SAMueL-1 at Editorial at: \url{https://doi.org/10.1161/STROKEAHA.122.039954}
    \item SAMueL-1 is used in, and cited, by the most recent national clinical guidelines for stroke care (Royal College of Physicians 2023). \textit{National Clinical Guideline for Stroke for the UK and Ireland} \url{https://www.strokeguideline.org/}).
    \item SAMueL is forming part of an NHS-England led initiative to improve thrombolysis use in England. This is the TASC (Thrombolysis in Acute Stroke Collaboration) team and will be launched in September 2023.
\end{itemize}

\subsubsection{Quotes}

\textbf{Geographical Modelling}

\textbf{Professor Tony Rudd}, National Clinical Director for Stroke, NHS England:

\textit{‘The modelling work undertaken by PenCHORD has been invaluable in helping the NHS decide how and where services for thrombectomy for stroke should be organised. It has also raised critical questions about the organisation of the whole of acute stroke care in way that will influence the new National Stroke Plan for England.’}

\textbf{Prof. Martin Dennis}, Specialty advisor to the Chief Medical Officer (Scotland):

\textit{`The modelling of hyperacute stroke care in Scotland was an important part of the Scottish Government’s programme to plan and implement a Scotland wide thrombectomy service. The modelling was commissioned by the Thrombectomy Advisory Group (TAG) and was based on data from the Scottish Stroke Care Audit. It provided estimates of the likely thrombolysis and thrombectomy volumes, their distributions across centres, and the likely gains in patient outcomes. The results of the modelling were important to the decisions by TAG to:}

\begin{enumerate}
    \item \textit{Emphasise the importance of optimising existing pathways for delivery of thrombolysis, to increase the numbers of patients treated and to reduce the door to needle times}

    \item {\textit{To advise, at least until pre-hospital diagnostic accuracy can be improved, to adopt a model where patients are transported by the ambulance service to the nearest hospital which can provide thrombolysis (drip) and only after the eligibility for thrombectomy has been determined, transported (ship) to a Thrombectomy hub.}}

    \item{\textit{Plan for a three thrombectomy centre model based in Glasgow, Edinburgh and Dundee which would collaborate to sustainably deliver thrombectomy in the middle of the night’}}
\end{enumerate}
 
\textbf{SAMueL}

\textbf{Noreen Kamal}, editor of \textit{Stroke}:

\textit{`Both machine learning and simulation modeling provide an exciting new frontier for acute stroke research. These [SAMueL] approaches can provide insight for process improvements that will result in better patient outcomes’}

\textbf{Dr. Richard Francis}, Head of Research, Stroke Association:

\textit{`We are very concerned by thrombolysis rates in England. Thrombolysis rates stagnated at around 11-12\% since 2013, and recently fell below 11\% for the first time in a decade. There is also considerable variation across the country. These figures are unacceptable. Big data and machine learning approaches like those used in this study give us hope. They offer exciting possibilities for improving stroke services. We are also pleased to see the researchers looking at what might be helpful for individual hospitals.’}

\textbf{Prof. Chris Price}, Professor of Stroke and Applied Research, Newcastle University: 

\textit{`These results are relevant both to individual clinicians and to systems. For instance, clinicians may focus more on establishing stroke onset time to make better treatment decisions. Services could set the target rate for thrombolysis at a level based upon evidence.The research could lead to changes in local service assessments and targets. It will also feed into other research I am supporting about emergency pathways of care for stroke.’}


\textbf{Dr. Jehath Syed}, Clinical Pharmacist and Researcher, JSS Academy of Higher Education \& Research, Mysuru: 

\textit{`Congratulations to the authors for this exhaustive work. It is the first of its kind and definitely adds value to the literature and practice. These findings will lead to further improvements to the healthcare system.’}

\textbf{Holly Maguire}, Advanced Nurse Practitioner for stroke, Mid Cheshire Hospital Trust: 

\textit{`I co-ordinate acute stroke training locally, and believe that some of these results will influence practice nationally. Simulation training is becoming more widely used with positive results across healthcare learning. Applying this learning to acute stroke treatment is an excellent advance. Providing robust training for emergency service provision is notoriously difficult. Prompt diagnosis and treatment is paramount and it can be difficult to provide services alongside training. The need to wait for someone to present with a particular condition can also make training a lengthy process. Simulation training could allow  standardised and robust training to be provided nationally, which would reduce intra-hospital variation.’
}
